\documentclass[../../main.tex]{subfiles}
\begin{document}
\subsection{Supraleiter}
\begin{itemize}
    \item sehr niedrige Temperatur => Widerstand = 0 (Supraleitfähigkeit)
    \item Der Leiter erreicht eine bestimmte Temperatur T(tiefergestelltes C) sprunghaft Null bzw. unmessbar klein
    \item Mehrere Tausend Legierungen und Verbindungen bekannt
    \item Supraleiter, Siedetemperatur > flüssigen Stickstoff (77 K) = Hochtemperatursupraleiter
    \item Die existierenden Supraleiter unterscheiden sich in ihrer Reaktion auf Magnetfelder 
\end{itemize}
\subsubsection{Beispiele}
\begin{itemize}
    \item Verlustfreie elektrische Energie durch supraleitende Kabel
    \item Magnetschwebetechnik/Magnetschwebebahn
    \item Magnetkameras für medizinische Untersuchungen (Kernspintomographie)
    \item Teilchenbeschleuniger
\end{itemize}

\todo{Modellvorstellung}
\end{document}