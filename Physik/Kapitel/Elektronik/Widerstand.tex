\documentclass[../../main.tex]{subfiles}
\begin{document}
\subsection{Widerstand}

\subsubsection{Modellvorstellung}
Die Leiter haben die unterschiedlichen Eigenschaften die Bewegung der freien Elektronen zu bremsen.
=> Sie haben unterschiedlichen Widerstand.

\subsubsection{Definition}
Widerstand = $\frac{Spannung}{Stromstärke}$ \newline
R = $\frac{U}{I}$ \newline
[R] = 1$\frac{V}{A}$ = 1 \si{\ohm} (Ohm) \newline

\subsubsection{Verhältnis von R zu l (Länge)}
Grafisch: \newline
    - R und l ergeben eine Ursprungshalbgerade \newline
rechnerisch: \newline
    - R ist direkt proportional zu l \newline
=> R ~ l

\subsubsection{Verhältnis von R zu A (Fläche)}
Grafisch: \newline
    - R und A ergeben einen Hyperbelast \newline
rechnerisch: \newline
    - R ist indirekt proportional zu A \newline
=> R ist indirekt proportional zu A

\subsubsection{spezifischer Widerstand}
$\si{\rho}$ = R * $\frac{A}{l}$ = $\frac{R * A}{l}$ \newline
[$\si{\rho}$] = 1 $\frac{\si{\ohm}{mm^2}}{m}$

\subsubsection{Bauformen von Widerständen}
\begin{itemize}
    \item Drahtwiderstand
    \item Schichtwiderstand
    \item SMD-Widerstände
    \item Schiebewiderstand (Potentiometer)
\end{itemize}

\subsection{Leitwert}
\subsubsection{Definition}
Widerstand = $\frac{Stromstärke}{Spannung}$ \newline
G = $\frac{I}{U}$ \newline
[G] = $\frac{1}{R}$ 1$\frac{A}{V}$ = 1 S (Siemens) \newline
\end{document}