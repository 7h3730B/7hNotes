\documentclass[../../main.tex]{subfiles}
\begin{document}
\subsection{Parallelschaltung}
\subsubsection{Stromstärke}
Alle Stromstärken addiert ergeben die Gesamtstromstärke.
($I_1 = \frac{U_1}{R_1}$)
\subsubsection{Spannung}
Die Spannungen sind überall gleich
\subsubsection{Widerstand}
Der Gesamtwiderstand ist immer kleiner als der kleinste Widerstand \\
$\frac{I_1}{I_2} = \frac{R_2}{R_1}$ \\
\\
$\frac{1}{R_G} = \frac{1}{R_1} + \frac{1}{R_2} ... => (\frac{1}{RG})^{-1} = RG$ \\
\subsubsection{Leitwert}
Alle Leitwerte addiert ergeben den Gesamtleitwert
\end{document}