\documentclass[../../main.tex]{subfiles}
\begin{document}
\subsection{Vorwiderstand}
\begin{itemize}
    \item Eine Reihenschaltung mit einem VOrwiderstand $R_v$ begrenzt den Strom beim Anschluss der Glühlampe an eine Spannungsquelle mit zu hoher Spannung.
    \item Ein Vorwiderstand führt auf jeden Fall zu Energieverlust
    \item Mit den Gesetzen der Reihenschaltung lässt sich der richtige Vorwiderstand berechnen
    \item Ein Schiebewiderstand wirkt als reiner Spannungsteiler
    \item Wenn ein Vorwiderstand im Spiel ist hat man immer einen Energieverlust in thermische Energie (Wärme)
\end{itemize}
\end{document}