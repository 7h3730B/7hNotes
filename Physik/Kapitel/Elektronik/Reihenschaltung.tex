\documentclass[../../main.tex]{subfiles}
\begin{document}
\subsection{Reihenschaltung}
\subsubsection{Stromstärke}
Die Stromstärke ist überall gleich.
(I = $\frac{U}{R}$)
\subsubsection{Spannung}
Alle Teilspannungen addiert ergeben die Gesamtspannung.
(U = R * I)
\subsubsection{Widerstand}
Alle Widerstände addiert ergeben den Gesamtwiderstand.
\\
\\
Die Spannungen stehen im selben Verhältnis wie die Widerstände
$\frac{U_G}{U_1} = \frac{R_G}{R_1}$
\end{document}