\documentclass[../../main.tex]{subfiles}
\begin{document}
\subsection{Wiederholung}
\subsubsection{physikalische Größen}
\begin{tabular}[h]{l|c|c|c}
physikalische Größe & Formelbuchstabe & Einheit \\
\hline
Stromstärke & I & 1A (Ampere) \\
Ladung & Q & 1C (Coulumb) \\
Spannung & U & 1V (Volt) \\
Leistung & P & 1W (Watt) ($\frac{J}{s}$) (V * A) \\
Widerstand & R & 1\si{\ohm} (Ohm) ($\frac{V}{A}$) \\
Siemens & G & 1S (Siemens) ($\frac{A}{V}$)
\end{tabular}
%------------
\subsubsection{Formeln}
\begin{itemize}
    \item R = $\frac{U}{I}$
    \item G = $\frac{I}{U}$
    \item P = U * I
    \item P = R * I²
    \item P = $\frac{U^2}{R}$
\end{itemize}
\end{document}