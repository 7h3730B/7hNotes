\documentclass[../Elektrizitaet.tex]{subfiles}
\begin{document}
free Electrons bewegen sich normalerweise zufällig, sie habne keine Richtung und auch keine Geschwindigkeit.  
Elektronen können aber so beeinflusst werden das sie sich koordiniert bewegen und so einen Strom Fluss darstellen.  
=> dynamic electricity / electricity  
Das Material schaut zwar für uns fest aus, hat aber in Wirklichkeit viel leeren Raum in sich und durch diesen Bewgen sich die Elektronen.

Das ganze kann man sich am besten mit Hilfe eines Kugelmodells vorstellen.
Stellen Sie sich eine Röhre vor, die bis oben hin mit Murmeln gefühlt ist, stecken sie nun oben eine Murmel hinein kommt unten sofort eine raus ohne das sich die einzelnen Murmeln weit bewegt haben.  
So ist es auch mit den Elektronen auch wenn das einschalten eines Stromkreises mit Lichtgeschwindigkeit passiert, bewegen sich die einzelnen Elektronen viel langsamer.
\end{document}